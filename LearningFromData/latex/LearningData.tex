%%%%%%%%%%%%%%%%%%%%%%%%%%%%%%%%%%%%%%%%%
% Arsclassica Article
% LaTeX Template
% Version 1.1 (10/6/14)
%
% This template has been downloaded from:
% http://www.LaTeXTemplates.com
%
% Original author:
% Lorenzo Pantieri (http://www.lorenzopantieri.net) with extensive modifications by:
% Vel (vel@latextemplates.com)
%
% License:
% CC BY-NC-SA 3.0 (http://creativecommons.org/licenses/by-nc-sa/3.0/)
%
%%%%%%%%%%%%%%%%%%%%%%%%%%%%%%%%%%%%%%%%%

%----------------------------------------------------------------------------------------
%	PACKAGES AND OTHER DOCUMENT CONFIGURATIONS
%----------------------------------------------------------------------------------------

\documentclass[
10pt, % Main document font size
a4paper, % Paper type, use 'letterpaper' for US Letter paper
oneside, % One page layout (no page indentation)
%twoside, % Two page layout (page indentation for binding and different headers)
headinclude,footinclude, % Extra spacing for the header and footer
BCOR5mm, % Binding correction
]{scrartcl}

\input{structure.tex} % Include the structure.tex file which specified the document structure and layout

\hyphenation{Fortran hy-phen-ation} % Specify custom hyphenation points in words with dashes where you would like hyphenation to occur, or alternatively, don't put any dashes in a word to stop hyphenation altogether

%----------------------------------------------------------------------------------------
%	TITLE AND AUTHOR(S)
%----------------------------------------------------------------------------------------

\title{\normalfont\spacedallcaps{Learing From Data}} % The article title

\author{\spacedlowsmallcaps{Xiaomo Liu}} % The article author(s) - author affiliations need to be specified in the AUTHOR AFFILIATIONS block

\date{} % An optional date to appear under the author(s)

%----------------------------------------------------------------------------------------

\begin{document}

%----------------------------------------------------------------------------------------
%	HEADERS
%----------------------------------------------------------------------------------------

\renewcommand{\sectionmark}[1]{\markright{\spacedlowsmallcaps{#1}}} % The header for all pages (oneside) or for even pages (twoside)
%\renewcommand{\subsectionmark}[1]{\markright{\thesubsection~#1}} % Uncomment when using the twoside option - this modifies the header on odd pages
\lehead{\mbox{\llap{\small\thepage\kern1em\color{halfgray} \vline}\color{halfgray}\hspace{0.5em}\rightmark\hfil}} % The header style

\pagestyle{scrheadings} % Enable the headers specified in this block

%----------------------------------------------------------------------------------------
%	TABLE OF CONTENTS & LISTS OF FIGURES AND TABLES
%----------------------------------------------------------------------------------------

\maketitle % Print the title/author/date block

\setcounter{tocdepth}{2} % Set the depth of the table of contents to show sections and subsections only

\tableofcontents % Print the table of contents

\listoffigures % Print the list of figures

\listoftables % Print the list of tables

%----------------------------------------------------------------------------------------
%	ABSTRACT
%----------------------------------------------------------------------------------------

\section*{Abstract} % This section will not appear in the table of contents due to the star (\section*)

\lipsum[1] % Dummy text

%----------------------------------------------------------------------------------------
%	AUTHOR AFFILIATIONS
%----------------------------------------------------------------------------------------

{\let\thefootnote\relax\footnotetext{* \textit{Department of Biology, University of Examples, London, United Kingdom}}}

{\let\thefootnote\relax\footnotetext{\textsuperscript{1} \textit{Department of Chemistry, University of Examples, London, United Kingdom}}}

%----------------------------------------------------------------------------------------

\newpage % Start the article content on the second page, remove this if you have a longer abstract that goes onto the second page

%----------------------------------------------------------------------------------------
%	INTRODUCTION
%----------------------------------------------------------------------------------------

\section{Perceptron Algorithm}
The perceptron algorithm is the most simple learning algorithm for classification problem. It works as follows
\begin{equation}
	h(x) = \mbox{sign} \left( \left(  \sum_{i=1}^{d}w_{i}x_{i} \right) + b \right)  
\end{equation}
where $w_{i}$ is the weight for feature $x_{i}$ and $b$ is a bias term. We can easily rewrite it in terms of linear algebra. We will treat bias $b$ as a constant weight $w_{0} = b$. With this convention $\mathbf{w}^{T} \mathbf{x}=\sum_{i=0}^{d} w_{i}x_{i}$. From this notation, it looks like there is no difference between weights and bias in perceptron model. A natural question is that if weight and bias are similar, why we even borther to introduce both weight and bias? 

\section{Logistic Regression}
The logistic regression utitilizes logistic function. 

\begin{equation}
	f(x) = \frac{L}{1 + e^{-k(x-x_{0})}}
\end{equation}
where $L$ defines the curve's maximum value and $k$ specifies steepness of the curve and $x_{0}$ is the centre of the curve. Sigmoid function is a special case of logistic function. The logistic regression is also a classification algorithm.

To answer this question, let take a look at the property of sigmoid function in Figure \ref{fig:sigmoid} and \ref{fig:sigmoid2}. In a nutshell, the weight $w$ controls its steepness of sigmoid functions and bias $b$ enables to shift the entire curve to left and right. Let take back and take glimpse at its properity.

\begin{itemize}
\item Sigmoid has a nearly linear range around its centre 0 and becomes non-linear at two ends.
\item Sigmoid has finite limits at negative infinity and infinity, most often going either from 0 to 1 or from -1 to 1 depending on convention.
\item Sigmoid can model many natural processes, such as those of complex system learning curves, exhibit a progression from small beginings that accelebrates and approaches a climax over time.

\item Sigmoid is differeniable. Its derivative is $\frac{1}{dx} y = y (1-y)$.
\end{itemize}

  
\begin{figure}
\centering
\includegraphics[width=0.8\textwidth]{Figures/sigmoid.pdf}
\caption{Sigmoid function with a single variant $x$, various choice of parameter $w$.}
\label{fig:sigmoid}
\end{figure}


\begin{figure}
\centering
\includegraphics[width=0.8\textwidth]{Figures/sigmoid2.pdf}
\caption{Sigmoid function with a single variant $x$, various choice of bias $b$.}
\label{fig:sigmoid2}
\end{figure}


The key difference between 

\section{Methods}



%----------------------------------------------------------------------------------------
%	BIBLIOGRAPHY
%----------------------------------------------------------------------------------------

\renewcommand{\refname}{\spacedlowsmallcaps{References}} % For modifying the bibliography heading

\bibliographystyle{unsrt}

\bibliography{sample.bib} % The file containing the bibliography

%----------------------------------------------------------------------------------------

\end{document}